\documentclass[hyperref={pdfpagelabels=false},compress,final]{beamer}
% \documentclass[hyperref={pdfpagelabels=false},compress,draft]{beamer}
%\let\Tiny=\tiny % get rid of some useless warnings...
\usepackage{lmodern} % also does the trick
%\usepackage{default}
\usepackage[english]{babel}
% some standard options
% \usepackage{verbatim}
\usepackage{graphicx} %If you want to include postscript graphics
\usepackage{graphics}
\usepackage{color}
\usepackage{epsfig}
\usepackage{minted}
\usepackage{url}
\usepackage{float}
%\usepackage{subfigure}
\usepackage{amsmath}
\usepackage{amsfonts}
\usepackage{amssymb}
\usepackage{longtable}
\usepackage{lipsum}
% \usepackage[latin1]{inputenc}
% \usepackage{times}
\usepackage{tikz}
\usetikzlibrary{arrows,shapes,fit}
%\usepackage{lineno}
%\usepackage[bf,nooneline,format=plain,indention=.3cm]{caption}
%\setlength{\captionmargin}{10pt}
%\usepackage[letterpaper,hmargin=1in,vmargin=1.25in]{geometry}
%\frenchspacing

% some tweaking for bibliography
\usepackage{natbib}
\renewcommand{\bibsection}{\subsubsection*{\bibname }}
\def\newblock{}

% local definitions
\newcommand{\foreign}[1]{{\it #1}}
\renewcommand{\emph}[1]{{\it #1}}

\newcommand{\myhw}{0.5\textwidth}
\newcommand{\myfigw}{0.9\textwidth}
\newcommand{\myw}{0.49\textwidth}
\newcommand{\lc}{life--cycle~}

% Equation references
\renewcommand{\eqref}[1]{Eq.~(\ref{#1})}
%\renewcommand{\vec}[1]{{\bf #1}}
% Abbreviations
\newcommand{\nn}{$N^2$}
\newcommand{\nnm}{$N^2_{max}$}
\newcommand{\zet}{$\zeta$}
\newcommand{\zetp}{$\zeta_{TP}$}
\newcommand{\dze}{$\Delta\zeta_{TP}$}
\newcommand{\htp}{$h_{TP}$}
\newcommand{\dttp}{$\Delta T_{TP}$}
\newcommand{\p}{$p$}
\renewcommand{\t}{$T$}
\renewcommand{\theta}{$\theta$}
% Directions
\newcommand{\degN}{\ensuremath{^\circ}N}
\newcommand{\degS}{\ensuremath{^\circ}S}
\newcommand{\degE}{\ensuremath{^\circ}E}
\newcommand{\degW}{\ensuremath{^\circ}W}
\newcommand{\degC}{\ensuremath{^\circ}C}
% other
\renewcommand{\deg}{\ensuremath{^\circ}} % use for degree in math mode
\newcommand{\st}{\ensuremath{^{st}}}
\newcommand{\nd}{\ensuremath{^{nd}}}
\newcommand{\rd}{\ensuremath{^{rd}}}
\renewcommand{\th}{\ensuremath{^{th}}}
\usepackage{verbatim} % the comment environment is in the verbatim package
% \renewenvironment{comment}{\begin{itemize}}{\end{itemize}}
\newcommand{\inlinecomment}[1]{{\noindent\sl#1}}
% \newcommand{\app}[1]{Appendix~\ref{#1}}
\newcommand{\app}[1]{Appendix~#1}
% Units (math environment)
\newcommand{\GB}{\ensuremath{\,\mbox{GB}}}
\newcommand{\ms}{\ensuremath{\,\mbox{m\;s}^{-1}}}
\newcommand{\mmday}{\ensuremath{\,\mbox{mm\;day}^{-1}}}
\newcommand{\kgmmday}{\ensuremath{\,\mbox{kg\;m}^{-2}\,\mbox{day}^{-1}}}
\newcommand{\kgmms}{\ensuremath{\,\mbox{kg\;m}^{-2}\,\mbox{s}^{-1}}}
\newcommand{\kgms}{\ensuremath{\,\mbox{kg\;m}^{-1}\,\mbox{s}^{-1}}}
\newcommand{\mss}{\ensuremath{\,\mbox{m\;s}^{-2}}}
\newcommand{\pvu}{\ensuremath{\,\mbox{PVU}}}
\newcommand{\kkm}{\ensuremath{\,\mbox{K\;km}^{-1}}}
\newcommand{\m}{\ensuremath{\,\mbox{m}}}
\newcommand{\s}{\ensuremath{\,\mbox{s}}}
\newcommand{\Kkm}{\ensuremath{\,\mbox{K\;km}^{-1}}}
\newcommand{\km}{\ensuremath{\,\mbox{km}}}
\newcommand{\K}{\ensuremath{\,\mbox{K}}}
\newcommand{\hPa}{\ensuremath{\,\mbox{hPa}}}
\newcommand{\mssq}{\ensuremath{\,\mbox{m\;s}^2}}
\newcommand{\JKmol}{\ensuremath{\,\mbox{J\;K}^{-1}\,\mbox{mol}^{-1}}}
\newcommand{\kgmol}{\ensuremath{\,\mbox{kg\;mol}^{-1}}}
\newcommand{\ssq}{\ensuremath{\times{}10^{-4}\,\mbox{s}^{-2}}}
% math-environment abbreviations
\newcommand{\ddz}[1]{\ensuremath{\frac{\partial #1}{\partial z}}}
\newcommand{\ddp}[1]{\ensuremath{\frac{\partial #1}{\partial p}}}

%
% New Environments
%
\newenvironment{myBox}[3][shadow=true]%
{\begin{center} \begin{minipage}{#2} \begin{beamerboxesrounded}[#1]{#3} \smallskip}%
      {\smallskip \end{beamerboxesrounded} \end{minipage} \end{center}}
% greenish box
\definecolor{lightgray}{rgb}{0.85,0.9,0.9}
\definecolor{darkgreen}{rgb}{0.0,0.4,0.2}
\setbeamercolor{upgreen}{fg=white,bg=darkgreen}
\setbeamercolor{lowgreen}{fg=black,bg=lightgray}
\newenvironment{myGreenBox}[3][upper=upgreen,lower=lowgreen,shadow=true]%
{\begin{center} \begin{minipage}{#2} \begin{beamerboxesrounded}[#1]{#3} \smallskip}%
      {\smallskip \end{beamerboxesrounded} \end{minipage} \end{center}}
% redish box
\definecolor{lightred}{rgb}{0.9,0.85,0.9}
\definecolor{darkred}{rgb}{0.5,0.0,0.2}
\setbeamercolor{upred}{fg=white,bg=darkred}
\setbeamercolor{lowred}{fg=black,bg=lightred}
\newenvironment{myRedBox}[3][upper=upred,lower=lowred,shadow=true]%
{\begin{center} \begin{minipage}{#2} \begin{beamerboxesrounded}[#1]{#3} \smallskip}%
      {\smallskip \end{beamerboxesrounded} \end{minipage} \end{center}}
%
% Hyphenation
\hyphenation{tro-po-pau-se}
\hyphenation{tro-po-pau-ses}
\hyphenation{pa-ra-me-ter-i-za-ti-ons}
\hyphenation{ra-di-o-son-de}
\hyphenation{me-ri-di-o-nal}
\hyphenation{a-na-ly-sis}
\hyphenation{ba-ro-cli-nic}
\hyphenation{geo-stro-phic}

%\DeclareMathOperator{\ln}{ln}
% increase line spacing by a factor of 1.5
%\renewcommand{\baselinestretch}{1.5}

\mode<presentation>{
\usetheme{Malmoe} % Malmoe Frankfurt
\usecolortheme{whale}
\usecolortheme{orchid}
\setbeamertemplate{bibliography item}[text]
\setbeamercovered{transparent}
}

\DeclareGraphicsExtensions{.png,.pdf,.jpg} %.pdf,.png,.jpg
\graphicspath{{figures/fullsize/}}
% \graphicspath{{figures/small/}}

\title[Large Ensembles with Python]{Managing Large Ensembles and Batch Execution with Python}
\subtitle{A Ensemble Class for Python}
\author[\href{http://www.physics.utoronto.ca/~aerler/}{Andre R. Erler} (\href{mailto:A.R.Erler@gmail.com}{A.R.Erler@gmail.com})]{Andre R. Erler}
\institute{PyCon Canada}
\date{November $12^{th}$, 2016}
%\logo{\includegraphics[scale=0.15]{HD3}}

% \AtBeginSection[]
% {
%    \begin{frame}{Outline}
%        \tableofcontents[currentsection]
%    \end{frame}
% }

% *************************************************************************************************
\providecommand\thispdfpagelabel[1]{}

\begin{document}

% \section*{\ }

\begin{frame}
\titlepage
\end{frame}

\begin{frame}{Outline}
\tableofcontents%[pausesections]
\end{frame}


% ################################################################################################


\section{Introduction}
% \begin{frame}{Outline}
% \tableofcontents[currentsection]
% \end{frame}


\subsection[Repetitive Science]{Science is Repetitive}

\begin{frame}{Science is Repetitive}
 \begin{columns}
   \begin{column}{0.6\textwidth}
     \small \medskip\\
     To reach conclusive results, scientific experiments usually have to be repeated many times; either to establish statistical significance, or to test a range of parameter values.\\ 
     \bigskip 
     \hspace*{-1cm}
     \includegraphics[width=1.2\textwidth]{GC_SampleTray}
   \end{column}
   \begin{column}{0.4\textwidth}
     \vspace*{-1.cm}\\%\hspace*{-0.5cm}
     \includegraphics[width=1.2\textwidth]{geo_samples}\\
    \bigskip 
     \small Experiments are planned and conducted in large batches or so-called \textit{ensembles}.\\ \smallskip
     \begin{myGreenBox}[shadow=true]{0.85\textwidth}{Automation}
       \small It is therefore desirable to automate the most repetitive tasks, and to create tools for this purpose.
     \end{myGreenBox}
  \end{column}
 \end{columns}
\end{frame}


\subsection*{What I do}

\begin{frame}{\hspace{0.55\textwidth} Coupling Climate Models\\ \hspace*{0.55\textwidth} with Hydrologic Models}
  \begin{columns}
    \begin{column}{0.5\textwidth}
      \vspace*{-1.5cm}
      \includegraphics[width=\textwidth]{Ts_annual_ortho_ndl}\\
      {\scriptsize Surface Temperature in a Global and a nested Regional Climate Model}
      \begin{myBox}{0.9\textwidth}{}
        \footnotesize I run Climate and Hydrologic Models to the impact of climate change on water resources and generate projections of future hydro-climate.
      \end{myBox}
    \end{column}
    \begin{column}{0.5\textwidth}
      \vspace*{.25cm}\\
      \scriptsize Athabasca River watershed: \\ groundwater depth (top) and surface water depth (bottom)
      \includegraphics[width=\textwidth]{ARB.png}
    \end{column}
  \end{columns}
\end{frame}

\begin{frame}{High Performance Computing}
  \begin{columns}[T]
 \begin{column}{0.5\textwidth}
  \begin{itemize}
  \item High-resolution Climate simulations:\\ \smallskip
    \begin{itemize}
     \item 4 days on 128 cores and 300GB of storage per model year
     \item 36 ensemble members, 15 years each
    \end{itemize}
  \medskip %\pause
  \item Surface-Subsurface Hydrologic Simulations:\\ \smallskip
    \begin{itemize}
     \item 1 day on 2 cores per model year
     \item also 15 years each, 100+ ensemble members
    \end{itemize}
  \end{itemize}
  %\medskip
  %\begin{myBox}{0.8\textwidth}{}
    %\footnotesize Run on SciNet HPC Facility %All simulations were performed on the SciNet super-computing facility, requiring more than $10^6$ core hours and 40\,TB storage.
  %\end{myBox}
 \end{column}
 \begin{column}{0.5\textwidth}
   \vspace*{-.8cm}
  \includegraphics[width=1.1\textwidth]{TCS-1}
 \end{column}
\end{columns}
\end{frame}


\subsection*{Numerical Experiments}

\begin{frame}{\hspace*{0.3cm} Numerical Experiments \\\hspace*{0.3cm} and Scripting}
  \begin{columns}
    \begin{column}{0.5\textwidth}
      \begin{itemize}
        \item<2-> The coupling process between GCM and RCM is ``off-line'' (asynchronous) \smallskip
        \item<2-> A pre-processing system converts GCM output into RCM (wrf-)input files \bigskip
        \item<3-> A RCM simulation is split into $\sim200$ separate jobs \smallskip
        \item<3-> The RCM runs continuously, each job submitting the next \bigskip
      \end{itemize}
    \end{column}
    \begin{column}{0.5\textwidth}
      \vspace*{-1.5cm}
      \begin{myBox}{0.9\textwidth}{The WRF Tools Package}
        \textbf{\color{teal}Python}
        \begin{itemize}
          \item<2> Run pre-processing\\ tool chain (WPS)
          \item<3> Initialize WRF jobs
          \item<4> Run post-processing \smallskip
        \end{itemize}
        {\color{purple}Shell Script}
        \begin{itemize}
          \item<2> Submit pre-processing
          \item<3> Run the WRF job,\\ submit next job
          \item<4> Archiving to tape \medskip
        \end{itemize}
        WRF Tools enables continuous and autonomous operation
      \end{myBox}
      %   WRF restarts from its own state, so the initialization from CESM is relatively irrelevant
    \end{column}
  \end{columns}
\end{frame}


% ################################################################################################


\section[\ Ensemble Class]{Batch Execution using an Ensemble Class}


\subsection*{Motivation}

\begin{frame}{\hspace*{0.3cm} Motivation \\\hspace*{0.3cm} Simplifyig Batch Execution}
  \begin{columns}
    \begin{column}{0.5\textwidth}
      \begin{itemize}
        \item<2-> The coupling process between GCM and RCM is ``off-line'' (asynchronous) \smallskip
        \item<2-> A pre-processing system converts GCM output into RCM (wrf-)input files \bigskip
        \item<3-> A RCM simulation is split into $\sim200$ separate jobs \smallskip
        \item<3-> The RCM runs continuously, each job submitting the next \bigskip
      \end{itemize}
    \end{column}
    \begin{column}{0.5\textwidth}
      \vspace*{-1.5cm}
      \begin{myBox}{0.9\textwidth}{The WRF Tools Package}
        \textbf{\color{teal}Python}
        \begin{itemize}
          \item<2> Run pre-processing\\ tool chain (WPS)
          \item<3> Initialize WRF jobs
          \item<4> Run post-processing \smallskip
        \end{itemize}
        {\color{purple}Shell Script}
        \begin{itemize}
          \item<2> Submit pre-processing
          \item<3> Run the WRF job,\\ submit next job
          \item<4> Archiving to tape \medskip
        \end{itemize}
        WRF Tools enables continuous and autonomous operation
      \end{myBox}
      %   WRF restarts from its own state, so the initialization from CESM is relatively irrelevant
    \end{column}
  \end{columns}
\end{frame}


\subsection[Ensemble Class]{The Ensemble Class}

\begin{frame}[fragile=singleslide]{\hspace*{0.3cm} Implementation\\\hspace*{0.3cm} using \texttt{\_\_getattr\_\_}}
  \begin{columns}
    \begin{column}{0.5\textwidth}
      Dataset/GCM specific parameters: \smallskip
      \begin{itemize}
        \item Input file types/names\smallskip
        \item Interpolation tables/grid \smallskip
        \item Variables / frequency \smallskip
      \end{itemize}
        \begin{myBox}{0.9\textwidth}{Multiple Datasets}
          \begin{itemize}
            \item \textit{\color{purple} Inheritance} for common procedures \smallskip
            \item \textit{\color{teal} Polymorphism} for different procedures
          \end{itemize}
        \end{myBox}
    \end{column}
    \begin{column}{0.5\textwidth}
      \vspace*{-1.25cm}
      \begin{myBox}{1.0\textwidth}{}
        %         \renewcommand{\theFancyVerbLine}{\sffamily\textcolor[rgb]{0.5,0.5,0.5}{\scriptsize\arabic{FancyVerbLine}}}
        \footnotesize
        \begin{minted}{python}
class Dataset(object):
  prefix = ''  # file prefix
  vtable = 'Vtable'
  gribname = 'GRIBFILE' # input
  ungrib_exe = 'ungrib.exe'
  ungrib_log = 'ungrib.exe.log'
  ...
  def __init__(self, ...):
    # type checking
    ...
  def setup(self, src, ...):
    ...
  def cleanup(self, tgt):
    ...
  def extractDate(self, fname):
    # match valid filenames
    ...
  def ungrib(self, date, mytag):
    # generate file for metgrid
    ...
        \end{minted}
      \end{myBox}
    \end{column}
  \end{columns}
\end{frame}


\subsection{Features}

% \begin{frame}{\hspace*{0.8cm} WPS: \textsc{\huge \color{purple} Fortran} \\\hspace*{0.8cm} Legacy Tools}
\begin{frame}{Additional Freatures}
  \begin{columns}
    \begin{column}{0.4\textwidth}
      \begin{myBox}{\textwidth}{WPS Components}
        \small
        \begin{enumerate}
          \item {\color{teal} \tt geogrid.exe} \\ static\,/\,geographic data
          \item {\color{red} \tt ungrib.exe / unccsm.exe} \\
          convert driving data to WRF IM Format
          \item {\color{blue} \tt metgrid.exe} \\ interpolate to WRF grid
          \item {\color{violet} \tt real.exe} \\ generate boundary condition files
        \end{enumerate}
      \end{myBox}
    \end{column}
    \begin{column}{0.5\textwidth}
      \textsc{\Large\color{purple} Fortran} legacy tools read from and write to temporary files:
      \begin{itemize}
        \item Strongly I/O limited in a HPC cluster environment
      \end{itemize}
      \medskip
      \onslide<2->{\begin{myBox}{0.8\textwidth}{The Solution (on Linux)}
        Run on RAM-disk!
        \begin{itemize}
          \item speedup $\sim\times10$
          \item requires 64\,GB RAM
        \end{itemize}
        Using Python driver script
      \end{myBox}}
    \end{column}
  \end{columns}
\end{frame}


% ################################################################################################


\section{Argument Expansion}


\subsection*{The Problem}

\begin{frame}[fragile=singleslide]{The Problem: Constructing Argument List}
  \begin{columns}
    \begin{column}{0.5\textwidth}
      \vspace*{-1.25cm}
      \begin{myBox}{1.0\textwidth}{}
        %         \renewcommand{\theFancyVerbLine}{\sffamily\textcolor[rgb]{0.5,0.5,0.5}{\scriptsize\arabic{FancyVerbLine}}}
        \footnotesize
        \begin{minted}{python}
class Dataset(object):
  prefix = ''  # file prefix
  vtable = 'Vtable'
  gribname = 'GRIBFILE' # input
  ungrib_exe = 'ungrib.exe'
  ungrib_log = 'ungrib.exe.log'
  ...
  def __init__(self, ...):
    # type checking
    ...
  def setup(self, src, ...):
    ...
  def cleanup(self, tgt):
    ...
  def extractDate(self, fname):
    # match valid filenames
    ...
  def ungrib(self, date, mytag):
    # generate file for metgrid
    ...
        \end{minted}
      \end{myBox}
    \end{column}
    \begin{column}{0.5\textwidth}
      Dataset/GCM specific parameters: \smallskip
      \begin{itemize}
        \item Input file types/names\smallskip
        \item Interpolation tables/grid \smallskip
        \item Variables / frequency \smallskip
      \end{itemize}
        \begin{myBox}{0.9\textwidth}{Multiple Datasets}
          \begin{itemize}
            \item \textit{\color{purple} Inheritance} for common procedures \smallskip
            \item \textit{\color{teal} Polymorphism} for different procedures
          \end{itemize}
        \end{myBox}
    \end{column}
  \end{columns}
\end{frame}


\section[Conclusion]{Concluding Remarks}

\subsection[Example]{The Ensemble in Operation}

\begin{frame}[fragile=singleslide]{The Ensemble in Operation}
  \begin{columns}
    \begin{column}{0.5\textwidth}
      Dataset/GCM specific parameters: \smallskip
      \begin{itemize}
        \item Input file types/names\smallskip
        \item Interpolation tables/grid \smallskip
        \item Variables / frequency \smallskip
      \end{itemize}
        \begin{myBox}{0.9\textwidth}{Multiple Datasets}
          \begin{itemize}
            \item \textit{\color{purple} Inheritance} for common procedures \smallskip
            \item \textit{\color{teal} Polymorphism} for different procedures
          \end{itemize}
        \end{myBox}
    \end{column}
    \begin{column}{0.5\textwidth}
      \vspace*{-1.25cm}
      \begin{myBox}{1.0\textwidth}{}
        %         \renewcommand{\theFancyVerbLine}{\sffamily\textcolor[rgb]{0.5,0.5,0.5}{\scriptsize\arabic{FancyVerbLine}}}
        \footnotesize
        \begin{minted}{python}
class Dataset(object):
  prefix = ''  # file prefix
  vtable = 'Vtable'
  gribname = 'GRIBFILE' # input
  ungrib_exe = 'ungrib.exe'
  ungrib_log = 'ungrib.exe.log'
  ...
  def __init__(self, ...):
    # type checking
    ...
  def setup(self, src, ...):
    ...
  def cleanup(self, tgt):
    ...
  def extractDate(self, fname):
    # match valid filenames
    ...
  def ungrib(self, date, mytag):
    # generate file for metgrid
    ...
        \end{minted}
      \end{myBox}
    \end{column}
  \end{columns}
\end{frame}


\subsection{Summary}

\begin{frame}{Summary \& Conclusion}

  {\Large \color{teal} \textbf{Python} }\smallskip
  \begin{itemize}
    \item Use Python for flow control (manage legacy tools) \medskip
    \item Parallelization relatively easy (within one node) \medskip
    \item Class structure is versatile and makes maintenance easier \medskip
  \end{itemize}
  \bigskip
  {\Large \color{purple} {RAM}-disk }\smallskip
  \begin{itemize}
    \item Scientific Programming: dealing with legacy tools\\
          Often in \textsc{Fortran}, often relying on disk I/O \medskip
    \item Use RAM-disk to avoid unnecessary disk I/O \medskip
  \end{itemize}
\end{frame}

\begin{frame}
 \vfill
 \begin{center}
 \LARGE
 Thank You!\hspace{1.5cm}  $\sim$ \hspace{1.5cm} Questions?
 \end{center}
 \vfill
\end{frame}


\end{document}
